\nonstopmode

\documentclass
[
    a4paper,
    twoside,
    12pt
]
{report}
\usepackage[utf8]{inputenc}
\renewcommand{\familydefault}{\rmdefault}
\usepackage[a4paper, left=3.19cm, right=3.19cm, top=2.54cm, bottom=2.54cm]{geometry}
\usepackage[american]{babel}
\usepackage{csquotes}
\usepackage{float}
\usepackage{enumerate}
\usepackage[bottom]{footmisc}
\usepackage{array}
\usepackage{ntheorem}
\usepackage{parskip}
\usepackage[right]{eurosym}
\usepackage{xcolor}
\usepackage[hyphens]{url}
\usepackage{makeidx}
\usepackage{multicol}
\usepackage{theorem}
\usepackage{listings}
\usepackage{graphicx}
\usepackage{pgfplots}
\pgfplotsset{compat=1.5}
\usepackage{csvsimple}
\usepackage{fancyhdr}
\usepackage{colortbl}
\usepackage{bchart}
\usepackage[hidelinks]{hyperref}
\usepackage{setspace}
\usepackage{mathptmx}
%\usepackage{showframe}
\pagestyle{plain}
\rhead{\thepage}
\sloppy

\usepackage[
backend=biber,
style=apa,
citestyle=authoryear
]{biblatex}

\addbibresource{references.bib}
\DeclareLanguageMapping{american}{american-apa}

%\setlength{\unitlength}{1cm}
%\setlength{\oddsidemargin}{0.3cm}
%\setlength{\evensidemargin}{0.3cm}
%\setlength{\textwidth}{15.5cm}
%\setlength{\topmargin}{-1.2cm}
%\setlength{\textheight}{24.7cm}
%\columnsep 0.5cm

%\title{Seminararbeit}
\selectcolormodel{gray}{

\newcommand{\Arbeitstitel}
	{
  	Success factors of video game consoles
	}
\newcommand{\Autor}
	{
		Dipl.-Ing. (FH) Lars Bartschat
	}

\newcommand{\MatrikelNr}
	{
		WBPA140000250
	}

\newcommand{\EmailAdresse}
	{
		bartschat@mailbox.org
	}
\newcommand{\Arbeitsart}
	{
		Seminar Paper
	}
\newcommand{\Studiengang}
	{
		Marketing Executive Program
	}
\newcommand{\Hochschule}
	{
		University of Münster
	}
\newcommand{\Lehrstuhl}
	{
		Department of Marketing and Media Research
	}
\newcommand{\Themensteller}
	{
		Prof. Dr. T. Hennig-Thurau
	}
\newcommand{\Betreuer}
	{
		M. Sc. R. Behrens
	}
\newcommand{\Ausgabedatum}
	{
		17.10.2016
	}
\newcommand{\Abgabedatum}
	{
		28.11.2016
	}
\newcommand{\Ort}
	{
		Münster}
			
		
\newcommand{\link}[1]{\ref{#1} (S. \pageref{#1})}
\begin{document}

\begin{titlepage}
    \vspace*{1.0cm}
    \begin{center}
        \begin{Large}
        \textbf{A neural network-based approach to accent conversion} \\
        \end{Large}
        \vspace*{1.0cm}
        \textit{Kenny W. Lino} \\
        \vspace*{1.5cm}
        Msc. Dissertation \\
        \vspace*{0.5cm}
        \begin{figure}[H]
        \centering
        \includegraphics[scale=0.15]{img/UM-coat-of-arms.png}
    	\end{figure}
       \vspace*{1.0cm}
       Department of Intelligent Computer Systems \\
       Faculty of Information and Communication Technology \\
       University of Malta \\
       2018 \\
       
       \vspace*{2.0cm}
	   Supervisors: \\
       Claudia Borg, Department of Artificial Intelligence, University of Malta \\
       Andrea De Marco, Institute of Space Sciences and Astronomy, University of Malta \\
       Eva Navas, Department of Communications Engineering, University of the Basque Country \\
       
       \vspace*{4.5cm}
       Submitted in partial fulfilment of the requirements for the Degree of \\
       European Master of Science in Human Language Science and Technology
    \end{center}
    

\end{titlepage}

\onehalfspacing
\pagenumbering{Roman}
\section*{Abstract}\addcontentsline{toc}{section}{Abstract}

With the emergence of the use of technology in language learning through
tools like Rosetta Stone and Duolingo, learners have slowly been given
more autonomy of their language learning projection. Although these
tools have allowed learners to tailor their learning to their own
liking, there is a gap between the available resources to assist those
that would like to improve their pronunciation. Previous research in the
intersection of language learning and speech technology has made efforts
to develop pronunciation training systems to address this problem, but
the systems themselves tend to have gaps due to the lack of appropriate
support for the users, especially in appropriately identifying errors
and providing sufficient feedback to help them correct their errors.

Some researchers have purported that alongside other forms of feedback
such as a visual articulatory representation, a voice conversion system
could serve as a potential feedback mechanism by helping learners
understand what their voice could sound like given the appropriate
changes. However, like pronunciation training systems, voice conversion
systems also faced many limitations especially in terms of the quality
which made them unrenderable as useful tools. With that said, recent
advances in speech technology using deep neural networks have become
increasingly successful in achieving better accuracy and quality in a
variety of tasks, allowing for the potential to return and address these
said gaps in quality and performance for voice conversion.

In this thesis, I aim to investigate these advancements in applying deep
neural networks to develop a voice conversion system that could
potentially serve as a feedback mechanism as a part of a larger
computer-based pronunciation training system. Specifically, I intend to
adapt the methodologies of Aryal and Gutizerrez-Osuna (2014) to set
forth an accent conversion system that strives to convert a source voice
into a target accent, leveraging neural network architectures in place
of Gaussian Mixture Models for conversion.
\cleardoublepage
\tableofcontents
\addcontentsline{toc}{section}{Contents}
\clearpage
\listoffigures
\addcontentsline{toc}{section}{List of Figures}

\section*{List of Abbreviations}\addcontentsline{toc}{section}{List of Abbreviations}\begin{tabular}{ll}
    CAPT    & Computer Assisted Pronunciation Training \\
    CP      & Critical Period \\
    L1/L2    & First and second language \\
    
\end{tabular}

\clearpage
\cleardoublepage
\pagenumbering{arabic} \setcounter{page}{1}

\chapter{Introduction}

Technology has continuously evolved to no bounds as witnessed by the
current successes enjoyed by the use of neural networks and the power of
current hardware, something perhaps predicted by Moore's Law who
proclaimed that computing power would double once every 18 months (and
then changed to 24 months) {[}CITE HERE{]}. We see the effects of neural
networks throughout many subareas in computer science, including that of
natural language processing. In fact, if we take a look at the number of
publications involving neural networks, it has exponentially compounded
annually {[}CITE IMAGE HERE{]}.

While technology has flourished and led to a number of new
state-of-the-art systems such as improvements in commercial speech
recognition and machine translation, it can be argued that these
benefits have not reached and innovated other areas to the same extent.
One such example is education. Although there have been small trends
here and there to create applications for educational use such as
Duolingo for language learning{[}EXAMPLES?{]}, in general it seems that
education has not evolved at the same rate. One particular example of
something that has been fairly stagnant in language education is
pronunciation. Unlike grammar and vocabulary, pronunciation can be
challenging to both learn and teach due to the lack of clarity on how to
teach it.

\section{Research Questions}\label{research-questions}

In this thesis, I focus on investigating the following questions:

\begin{itemize}
\item
  How can we leverage deep neural network technology and voice
  conversion to convert language learner's speech into sounding more
  native-like?
\item
  Should we be able to create a sound voice conversion system, would it
  be possible to convert the language learner's speech with minimal
  (non-parallel) audio?
\end{itemize}

\section{Thesis Overview}\label{thesis-overview}

The overview of the thesis is as follows: The main research question of
this thesis is the following:
\section{Background and related work}\label{background-and-related-work}

In this section, I provide a brief overview of second language
acquisition and education in order to motivate the usage of technology
in language learning using tools such as the one proposed here in this
thesis. I then examine some previous research in computer assisted
pronunciation (CAPT) systems in order to frame the successes and gaps of
such work, and close with a discussion about voice conversion and accent
conversion.

\subsection{Theoretical and educational
motivations}\label{theoretical-and-educational-motivations}

\label{sec:theo-edu} Linguists have long debated over the possibility of
whether second language (L2) learners (e.g.~adult learners) could ever
acquire a language to the extent of a native speaker. Some still cite
ideas like the Critical Period (CP) Hypothesis and neuroplasticity which
claims that learners cannot acquire language (at least as well as a
native speaker) after a certain point in time due to the loss of
plasticity in the brain \parencite{lenneberg1967,scovel1988}. This
theory has been particularly cited in reference to pronunciation,
perhaps due to the obvious difficultly in overcoming the L1 negative
transfer that many, if not all, language learners experience in speaking
a new language.

Since the emergence of the CP hypothesis, many researchers have come to
find evidence that suggest the contrary. In \textcite{lengeris2012}, we
are presented an overview of the interactions between factors that
affect second language acquisition such as age, linguistic experience,
and learning setting. Here, we find evidence of studies such as
\textcite{bongaerts1995}, which present a counterargument against the CP
hypothesis. In this study, they discovered through a foreign accent
rating study with Dutch learners of English that learners could be
perceived as \textit{indistinguishable} from native speakers. Other
researchers such as Flege have also found that there is no distinct
`cut-off' point like the CP suggests. Thus, while age may have some
effect on a speaker's pronunciation, there is no conclusive evidence to
say that the loss of plasticity in the brain leads to an inability to
acquire language. As \textcite{lengeris2012} states, evidence for the CP
hypothesis would require `a sharp drop-off in a learner's abilities',
and `all early L2 learners should achieve native-like performance' (and
vice versa). This is not to say that learners are not still deterred by
other aspects like their own L1, but this does highlight the potential
that learners could be taught pronunciation, given the right settings.

Aside from the issue of whether or not language learners could ever
achieve native-like performance, another question that arises is whether
or not there is even a \textit{need} for learners to aim so high. In
\textcite{munro1999}, they discuss the interaction between foreign
accent, comprehensibility and intelligibility and point out that the
goal for many L2 learners is to communicate and not necessarily sound
like a native speaker. They also conduct a study to prove that despite
the fact that some speakers may have what some consider a `heavy
accent', that this does not automatically mean that they are
unintelligible. They found in their study that errors in prosody tended
to affect the speakers' intelligibility the most, which underscores the
role of prosody in organizing our utterances.

While linguists make these discoveries and observations of L2 learning,
it seems that it takes a lot of effort for them to trickle down to the
foreign language classroom. In \textcite{darcy2012}, they find through a
small survey of 14 teachers that although teachers tend to find
pronunciation to be `very important', the majority do not teach it at
all. When asked why they do not teach it, they cited reasons such as
`time, a lack of training and the need for more guidance and
institutional support'. Even though the number of teachers surveyed may
be significantly small, this gives us a glimpse through the lens of what
language teachers themselves experience in relation to pronunciation. We
see that even though teachers would like to address it, this would
require a restructuring in their curriculum and training-- something
that would undoubtedly take even more time before students get more
pronunciation attention. Compounded with the issue of time and the fact
that not all learners need or want equal amount of pronunciation
training, it may be unlikely to see such change in second language
curriculum so soon.

This points to the potential solution of employing a technology-based
system to improve pronunciation as learners could individually address
their needs \textit{outside} of the classroom.

\subsection{Computer-assisted pronunciation training
systems}\label{computer-assisted-pronunciation-training-systems}

\label{sec:capt} With the improvements of technology and speech
processing, researchers have attempted to make a number of
computer-assisted pronunciation training (CAPT) systems. In general,
CAPT systems utilize some form of automatic speech recognition (ASR) to
record a speaker and compares their recordings (usually) with a native
speaker gold standard. They also usually include a feedback mechanism
with a combination of pitch contours, spectrograms or audio recordings
to help the user adjust their pronunciation.

In \textcite{neri2002}, we are presented with an overview of the
interaction between language pedagogy and CAPT systems. Here, we see
that aside from the classroom, there seems to be an issue in relating
the findings of linguistics/language pedagogy with technology. Part of
the reason, they suggest, stems from the fact that there are not `clear
guidelines' on how to adapt second language acquisition research and
thus many CAPT systems `fail to meet sound pedagogical requirements'.
They emphasize the need for the learners to have appropriate input,
output, and feedback and exhibit how the systems available at the time
were lacking. For example, they criticize some CAPT systems that were
prevalent at the time including systems like \textit{Pro-nunciation} and
the \textit{Tell Me More} series for utilizing feedback systems that
give the users feedback in waveforms and spectrograms, which cannot be
easily interpreted without training. Further, they argue that although
visual feedback has its merits, this kind of feedback suggests to the
user that their utterance must look close to what is shown on the
screen, which is not the case. An utterance can be pronounced perfectly
fine, but look completely different from a spectrogram, and
\textit{especially} a waveform due to the number of features represented
in each visualization, such as the intensity, which will indefinitely
vary from user to user and the given examplar. They conclude their
article by making it a point to discuss recommendations for CAPT
systems, by stating that they should integrate what has been found in
research from second language acquisition, and to train pronunciation in
a communicative manner to give context to the learners. They also point
to the problematic area of feedback and advise that systems provide more
easily interpretable feedback with both audio and visual information,
and propose that systems give exercises that are `realistic, varied, and
engaging'. Despite the fact that this article was published in 2002,
this article provides a sound basis in addressing the proper makings of
a successful CAPT system.

In another article by \textcite{eskenazi2009}, we are given a brief
review of technologies in CAPT systems, this time more focused from a
technical perspective. In particular, she gives attention to the
different CAPT system types and provides information on prosody
detection and complete tutoring systems.

She first explains that CAPT systems can be generally split into two
main types: individual error detection and pronunciation assessment. As
indicated, individual error detection systems are more focused on one
particular aspect of the user's speech, such as the phones or pitch,
while pronunciation assessment systems are more designed to represent
how a human would judge a non-native utterance.

Early individual error detection systems, including one of her very own
\textcite{eskenazi1998}, started by using a variety of speech
recognition techniques such as forced alignment or unconstrained speech
recognition. They also worked with a variety of measures to detect the
differences between the individual errors and gold standard. Some of
these measures include hidden Markov model (HMM) based recognition
scoring, a confidence score based system known as Goodness of
Pronunciation (GOP), and Linear Discriminant Analysis (LDA). Each of
these measures were found to somehow detect the users' errors; however
they suffer from issues like low precision or the need for a very
homogeneous sample (e.g.~Japanese speakers).

Here, \textcite{eskenazi2009} makes a point that working to improve
non-native pronunciation is not simply a binary question of native
vs.~non-native; instead the L1 of the system's users must be considered,
as this can greatly affect the evaluation. She also points out that the
level of language learning of the speakers can also impact the metrics
and success of the system as well, and thus an appropriate population
must be selected carefully when building a CAPT system, especially when
considering individual errors.

In her discussion of prosody correction, she points to pivotal works
that have used a variety of manners to address the issue. Some works
include systems that use Pitch Synchronous Overlap and Add (PSOLA) to
resynthesize the prosody of users to help them hear what an appropriate
utterance would sound like. This in particular could be a potentially
effective feedback mechanism to employ in future systems, as it has been
said that imitating one's own voice is the most effective. Other systems
she mentions include systems that use appropriate L2 phonological models
and break prosody down into two levels--- syllable-word and
utterance-phrase, and systems that detect the `liveliness' of a speaker.
However, she does not discuss prosody correction systems in much detail,
which may suggest that there is not as much research in this particular
area as compared to the individual error systems. Regardless, these
works all provide interesting paths to consider in developing a prosody
correction system.

Similar to \textcite{eskenazi2009}, \textcite{chun2008}, presents a
review of various technologies, this time related directly to prosody.
They discuss four main tools in teaching prosody: `visualization of
pitch contours', `multimodal tools', `spectrographic displays' and
`vowel analysis programs'. Citing previous work, it appears that they
suggest that the visualization of pitch contours is the most robust
method of feedback for learners as it is the most intuitive and
non-language specific. Aside from this however, they also discuss the
potential of a multimedia approach used by \textcite{hardison2005} that
integrate both audio and video in a system called \textit{Anvil}.
Following this research, users of this system were able to generalize
their training beyond a sentence level and were able to perform better
at a discourse-level. This again emphasizes the point that prosody
training should put the language in context, which is an important
aspect to consider prosody training, as we know how prosody works in
relation to communication.

They also discuss the two main methods of such prosody systems: one
which utilizes isolated scripted sentences and the other utilizing
imitation. They conclude that neither method is useful for generalizing
to novel methods and suggest that the training should relate to the
ultimate goal. Other information they provide in this article are
prosody models used in previous studies. We see that some previous
studies have focused on utilizing a variety of sentence types to teach
prosody, contrasting \textit{wh}-questions, echo questions, either-or
questions and statements. Like the other articles, the works examined in
\textcite{chun2008} gives us insight on potential ways to improve future
CAPT systems, as we are shown exemplars of potential input and positive
reinforcements in successful types of feedback for the user. They
conclude that in order to create better pronunciation training systems,
we should take advantage of recent technology.

\subsection{Voice conversion}\label{voice-conversion}

Similar to CAPT systems, many researchers have steadily progressed on
building voice conversion systems. However, unlike CAPT systems, VC
systems are not typically designed for uses in language learning and are
more grounded in other speech technology uses such as text-to-speech
synthesis.

To properly frame voice conversion, we take a look at
\cite{mohammadi2017} who presents a recent overview of the subfield.
Following a definition setforth by the authors, voice conversion refers
to the transformation of a speech signal of a \emph{target speaker} to
make it sound similar to a \emph{source speaker} in any chosen fashion
with the utterance still being intact \cite{mohammadi2017}. Some of
these changes can include changes in emotion, accent, or phonation
(whispered/murmured speech). there have been a number of proposed uses
for VC, including the transformation of speaker identity (perhaps for
voice dubbing), personalized TTS systems, and against biometric voice
authentication systems.

Voice conversion often involves a large number of processes, one of
which includes deciding the appropriate type of data. To start, one must
decide whether to have parallel or non-parallel speech data. Parallel
speech data refers to speech data that has source and reference speakers
that say the same utterance, so only the speaker information is
different, while non-parallel data would indicate datasets where the
utterances are not the same.

Even though work in VC has progressed within the last decade or so, it
seems that the task still presents a large challenge for researchers due
to the various nuanced steps and features required to have high quality
voice conversion. This can be witnessed for example, in a shared task
dedicated to voice conversion, appropriately called \emph{The Voice
Conversion Challenge} where many research groups involved in speech
technology around the world have submitted systems in attempts to tackle
the issue. In the second iteration of the challenge
\cite{lorenzo-trueba2018}, the organizers proposed both a parallel and
non-parallel version of the task, both of which were evaluated on
natural and similarity using crowdsourcing.

The type of systems submitted to this year's version of the task
displays the current state of the field and perhaps machine learning
research in general as this year saw a huge increase in the number of
systems using neural networks. However, it does not go without saying
that there were indeed systems that used more traditional statistical
methods, such as Gaussian Mixture Models (GMM) and one of its
variations, differential GMM (DIFFGMM).

In order to evaluate the systems, a group of listeners roughly 300 were
gathered using crowdsourcing. We

Thus, even though not many systems were neural network based, only one
neural network based system was able to outperform the sprocket
GMM-based baseline.

Although we see limitations in the systems presented in The 2018 Voice
Conversion Challenge, there have also been some incredible breakthroughs
in systems set forth by research teams at Google Brain. One such system
involves the Tacotron end-to-end system, which has been proposed to
replace the current set-up of text-to-speech systems by reducing the
amount of components (decoder, vocoder, etc. {[}IS THIS TRUE?{]}) into
one piece. The researchers working on this system have recently revealed
a impressive system that also takes advantage of deep neural networks to
encode speaker characteristics into embeddings, which are then utilized
to transfer style \parencite{wang2018}.

With that said, it is evident that the reason for the success of their
systems is due to the availability of large-scale, high quality data
that many research institutions do not have access to or have funding
for. Thus, it may be a long while before the general public has the
ability to replicate such systems; however it is extremely exciting to
know that there is the possibility.

\subsection{Accent conversion}\label{accent-conversion}

Like voice conversion, accent conversion is dedicated to convert the
speech of a \emph{target speaker} into sounding more like a \emph{source
speaker}. However, accent conversion is specifically focused on morphing
the \emph{accent} of the speech signal, as opposed to sounding directly
like the source speaker. Succinctly stated, ``Accent conversion seeks to
transform second language L2 utterances to appear as if produced with a
native (L1) accent,'' \parencite{aryal2014a}. Accent conversion poses a
further challenge on top of (parallel) voice conversion as the audio of
the source speaker and target speaker is often forced-aligned. This
means that with native and non-native speech, voice conversion would
retain the voice quality and accent of the target speaker
\parencite{aryal2014}.

Finding previous work done in accent conversion has proven to be a
difficult task as there are not many articles available in the area;
this may be because work in voice conversion itself is already a
subfield of speech technology.

With that said, \cite{aryal2014} and other works done by the group of
researchers have made efforts to address the challenge. Throughout their
research, they test a variety of methodologies, including accent
conversion through voice morphing and articulatory synthesis. In the
same work of \cite{aryal2014}, they propose a variation to standard
forced alignment techniques used in voice conversion to pair frames
based on acoustic similarity. To do achieve this, they first dampen
vocal tract differences between the speakers using a method known as
\emph{vocal tract length normalization}.

Finally, in \cite{aryal2015}, they also join in on the rising trend of
utilizing deep neural networks. However, instead of training on solely
audio, they utilize articulatory information for real-time conversion.

Evidentally, an articulatory-based accent conversion system requires
specialized technology that is inaccessible to most users, and thus such
a system cannot be adopted easily.

Aside from the work done by these researchers, not much has been done
since to address accent conversion. Looking at their recent
publications, it seems that they have also halted work in this area, as
they have not published articles in the area since 2016; thus this
leaves a gap between the potential for accent conversion and language
learners.
\cleardoublepage
% \bibliographystyle{apalike-url}
% \renewcommand\bibname{References}
% \nocite{*}
% \bibliography{references}

\printbibliography
\end{document}
\nonstopmode

\documentclass
[
    a4paper,
    twoside,
    12pt
]
{report}
\usepackage[utf8]{inputenc}
\renewcommand{\familydefault}{\rmdefault}
\usepackage[a4paper, left=3.19cm, right=3.19cm, top=2.54cm, bottom=2.54cm]{geometry}
\usepackage[american]{babel}
\usepackage{csquotes}
\usepackage{float}
\usepackage{enumerate}
\usepackage[bottom]{footmisc}
\usepackage{array}
\usepackage{ntheorem}
\usepackage{parskip}
\usepackage[right]{eurosym}
\usepackage{xcolor}
\usepackage[hyphens]{url}
\usepackage{makeidx}
\usepackage{multicol}
\usepackage{theorem}
\usepackage{listings}
\usepackage{graphicx}
\usepackage{pgfplots}
\pgfplotsset{compat=1.5}
\usepackage{csvsimple}
\usepackage{fancyhdr}
\usepackage{colortbl}
\usepackage{bchart}
\usepackage[hidelinks]{hyperref}
\usepackage{setspace}
\usepackage{mathptmx}
%\usepackage{showframe}
\pagestyle{plain}
\rhead{\thepage}
\sloppy

\usepackage[
backend=biber,
style=apa,
citestyle=authoryear
]{biblatex}

\addbibresource{references.bib}
\DeclareLanguageMapping{american}{american-apa}

%\setlength{\unitlength}{1cm}
%\setlength{\oddsidemargin}{0.3cm}
%\setlength{\evensidemargin}{0.3cm}
%\setlength{\textwidth}{15.5cm}
%\setlength{\topmargin}{-1.2cm}
%\setlength{\textheight}{24.7cm}
%\columnsep 0.5cm

%\title{Seminararbeit}
\selectcolormodel{gray}{

\newcommand{\Arbeitstitel}
	{
  	Success factors of video game consoles
	}
\newcommand{\Autor}
	{
		Dipl.-Ing. (FH) Lars Bartschat
	}

\newcommand{\MatrikelNr}
	{
		WBPA140000250
	}

\newcommand{\EmailAdresse}
	{
		bartschat@mailbox.org
	}
\newcommand{\Arbeitsart}
	{
		Seminar Paper
	}
\newcommand{\Studiengang}
	{
		Marketing Executive Program
	}
\newcommand{\Hochschule}
	{
		University of Münster
	}
\newcommand{\Lehrstuhl}
	{
		Department of Marketing and Media Research
	}
\newcommand{\Themensteller}
	{
		Prof. Dr. T. Hennig-Thurau
	}
\newcommand{\Betreuer}
	{
		M. Sc. R. Behrens
	}
\newcommand{\Ausgabedatum}
	{
		17.10.2016
	}
\newcommand{\Abgabedatum}
	{
		28.11.2016
	}
\newcommand{\Ort}
	{
		Münster}
			
		
\newcommand{\link}[1]{\ref{#1} (S. \pageref{#1})}
\begin{document}

\begin{titlepage}
    \vspace*{1.0cm}
    \begin{center}
        \begin{Large}
        \textbf{A neural network-based approach to accent conversion} \\
        \end{Large}
        \vspace*{1.0cm}
        \textit{Kenny W. Lino} \\
        \vspace*{1.5cm}
        Msc. Dissertation \\
        \vspace*{0.5cm}
        \begin{figure}[H]
        \centering
        \includegraphics[scale=0.15]{img/UM-coat-of-arms.png}
    	\end{figure}
       \vspace*{1.0cm}
       Department of Intelligent Computer Systems \\
       Faculty of Information and Communication Technology \\
       University of Malta \\
       2018 \\
       
       \vspace*{2.0cm}
	   Supervisors: \\
       Claudia Borg, Department of Artificial Intelligence, University of Malta \\
       Andrea De Marco, Institute of Space Sciences and Astronomy, University of Malta \\
       Eva Navas, Department of Communications Engineering, University of the Basque Country \\
       
       \vspace*{4.5cm}
       Submitted in partial fulfilment of the requirements for the Degree of \\
       European Master of Science in Human Language Science and Technology
    \end{center}
    

\end{titlepage}

\onehalfspacing
\pagenumbering{Roman}
\section*{Abstract}\addcontentsline{toc}{section}{Abstract}

With the emergence of the use of technology in language learning through
tools like Rosetta Stone and Duolingo, learners have slowly been given
more autonomy of their language learning projection. Although these
tools have allowed learners to tailor their learning to their own
liking, there is a gap between the available resources to assist those
that would like to improve their pronunciation. Previous research in the
intersection of language learning and speech technology has made efforts
to develop pronunciation training systems to address this problem, but
the systems themselves tend to have gaps due to the lack of appropriate
support for the users, especially in appropriately identifying errors
and providing sufficient feedback to help them correct their errors.

Some researchers have purported that alongside other forms of feedback
such as a visual articulatory representation, a voice conversion system
could serve as a potential feedback mechanism by helping learners
understand what their voice could sound like given the appropriate
changes. However, like pronunciation training systems, voice conversion
systems also faced many limitations especially in terms of the quality
which made them unrenderable as useful tools. With that said, recent
advances in speech technology using deep neural networks have become
increasingly successful in achieving better accuracy and quality in a
variety of tasks, allowing for the potential to return and address these
said gaps in quality and performance for voice conversion.

In this thesis, I aim to investigate these advancements in applying deep
neural networks to develop a voice conversion system that could
potentially serve as a feedback mechanism as a part of a larger
computer-based pronunciation training system. Specifically, I intend to
adapt the methodologies of Aryal and Gutizerrez-Osuna (2014) to set
forth an accent conversion system that strives to convert a source voice
into a target accent, leveraging neural network architectures in place
of Gaussian Mixture Models for conversion.
\cleardoublepage
\tableofcontents
\addcontentsline{toc}{section}{Contents}
\clearpage
\listoffigures
\addcontentsline{toc}{section}{List of Figures}

\section*{List of Abbreviations}\addcontentsline{toc}{section}{List of Abbreviations}\begin{tabular}{ll}
    CAPT    & Computer Assisted Pronunciation Training \\
    CP      & Critical Period \\
    L1/L2    & First and second language \\
    
\end{tabular}

\clearpage
\cleardoublepage
\pagenumbering{arabic} \setcounter{page}{1}

\chapter{Introduction}

Technology has continuously evolved to no bounds as witnessed by the
current successes enjoyed by the use of neural networks and the power of
current hardware, something perhaps predicted by Moore's Law who
proclaimed that computing power would double once every 18 months (and
then changed to 24 months) {[}CITE HERE{]}. We see the effects of neural
networks throughout many subareas in computer science, including that of
natural language processing. In fact, if we take a look at the number of
publications involving neural networks, it has exponentially compounded
annually {[}CITE IMAGE HERE{]}.

While technology has flourished and led to a number of new
state-of-the-art systems such as improvements in commercial speech
recognition and machine translation, it can be argued that these
benefits have not reached and innovated other areas to the same extent.
One such example is education. Although there have been small trends
here and there to create applications for educational use such as
Duolingo for language learning{[}EXAMPLES?{]}, in general it seems that
education has not evolved at the same rate. One particular example of
something that has been fairly stagnant in language education is
pronunciation. Unlike grammar and vocabulary, pronunciation can be
challenging to both learn and teach due to the lack of clarity on how to
teach it.

\section{Research Questions}\label{research-questions}

In this thesis, I focus on investigating the following questions:

\begin{itemize}
\item
  How can we leverage deep neural network technology and voice
  conversion to convert language learner's speech into sounding more
  native-like?
\item
  Should we be able to create a sound voice conversion system, would it
  be possible to convert the language learner's speech with minimal
  (non-parallel) audio?
\end{itemize}

\section{Thesis Overview}\label{thesis-overview}

The overview of the thesis is as follows: The main research question of
this thesis is the following:
\section{Background and related work}\label{background-and-related-work}

In this section, I provide a brief overview of second language
acquisition and education in order to motivate the usage of technology
in language learning using tools such as the one proposed here in this
thesis. I then examine some previous research in computer assisted
pronunciation (CAPT) systems in order to frame the successes and gaps of
such work, and close with a discussion about voice conversion and accent
conversion.

\subsection{Theoretical and educational
motivations}\label{theoretical-and-educational-motivations}

\label{sec:theo-edu} Linguists have long debated over the possibility of
whether second language (L2) learners (e.g.~adult learners) could ever
acquire a language to the extent of a native speaker. Some still cite
ideas like the Critical Period (CP) Hypothesis and neuroplasticity which
claims that learners cannot acquire language (at least as well as a
native speaker) after a certain point in time due to the loss of
plasticity in the brain \parencite{lenneberg1967,scovel1988}. This
theory has been particularly cited in reference to pronunciation,
perhaps due to the obvious difficultly in overcoming the L1 negative
transfer that many, if not all, language learners experience in speaking
a new language.

Since the emergence of the CP hypothesis, many researchers have come to
find evidence that suggest the contrary. In \textcite{lengeris2012}, we
are presented an overview of the interactions between factors that
affect second language acquisition such as age, linguistic experience,
and learning setting. Here, we find evidence of studies such as
\textcite{bongaerts1995}, which present a counterargument against the CP
hypothesis. In this study, they discovered through a foreign accent
rating study with Dutch learners of English that learners could be
perceived as \textit{indistinguishable} from native speakers. Other
researchers such as Flege have also found that there is no distinct
`cut-off' point like the CP suggests. Thus, while age may have some
effect on a speaker's pronunciation, there is no conclusive evidence to
say that the loss of plasticity in the brain leads to an inability to
acquire language. As \textcite{lengeris2012} states, evidence for the CP
hypothesis would require `a sharp drop-off in a learner's abilities',
and `all early L2 learners should achieve native-like performance' (and
vice versa). This is not to say that learners are not still deterred by
other aspects like their own L1, but this does highlight the potential
that learners could be taught pronunciation, given the right settings.

Aside from the issue of whether or not language learners could ever
achieve native-like performance, another question that arises is whether
or not there is even a \textit{need} for learners to aim so high. In
\textcite{munro1999}, they discuss the interaction between foreign
accent, comprehensibility and intelligibility and point out that the
goal for many L2 learners is to communicate and not necessarily sound
like a native speaker. They also conduct a study to prove that despite
the fact that some speakers may have what some consider a `heavy
accent', that this does not automatically mean that they are
unintelligible. They found in their study that errors in prosody tended
to affect the speakers' intelligibility the most, which underscores the
role of prosody in organizing our utterances.

While linguists make these discoveries and observations of L2 learning,
it seems that it takes a lot of effort for them to trickle down to the
foreign language classroom. In \textcite{darcy2012}, they find through a
small survey of 14 teachers that although teachers tend to find
pronunciation to be `very important', the majority do not teach it at
all. When asked why they do not teach it, they cited reasons such as
`time, a lack of training and the need for more guidance and
institutional support'. Even though the number of teachers surveyed may
be significantly small, this gives us a glimpse through the lens of what
language teachers themselves experience in relation to pronunciation. We
see that even though teachers would like to address it, this would
require a restructuring in their curriculum and training-- something
that would undoubtedly take even more time before students get more
pronunciation attention. Compounded with the issue of time and the fact
that not all learners need or want equal amount of pronunciation
training, it may be unlikely to see such change in second language
curriculum so soon.

This points to the potential solution of employing a technology-based
system to improve pronunciation as learners could individually address
their needs \textit{outside} of the classroom.

\subsection{Computer-assisted pronunciation training
systems}\label{computer-assisted-pronunciation-training-systems}

\label{sec:capt} With the improvements of technology and speech
processing, researchers have attempted to make a number of
computer-assisted pronunciation training (CAPT) systems. In general,
CAPT systems utilize some form of automatic speech recognition (ASR) to
record a speaker and compares their recordings (usually) with a native
speaker gold standard. They also usually include a feedback mechanism
with a combination of pitch contours, spectrograms or audio recordings
to help the user adjust their pronunciation.

In \textcite{neri2002}, we are presented with an overview of the
interaction between language pedagogy and CAPT systems. Here, we see
that aside from the classroom, there seems to be an issue in relating
the findings of linguistics/language pedagogy with technology. Part of
the reason, they suggest, stems from the fact that there are not `clear
guidelines' on how to adapt second language acquisition research and
thus many CAPT systems `fail to meet sound pedagogical requirements'.
They emphasize the need for the learners to have appropriate input,
output, and feedback and exhibit how the systems available at the time
were lacking. For example, they criticize some CAPT systems that were
prevalent at the time including systems like \textit{Pro-nunciation} and
the \textit{Tell Me More} series for utilizing feedback systems that
give the users feedback in waveforms and spectrograms, which cannot be
easily interpreted without training. Further, they argue that although
visual feedback has its merits, this kind of feedback suggests to the
user that their utterance must look close to what is shown on the
screen, which is not the case. An utterance can be pronounced perfectly
fine, but look completely different from a spectrogram, and
\textit{especially} a waveform due to the number of features represented
in each visualization, such as the intensity, which will indefinitely
vary from user to user and the given examplar. They conclude their
article by making it a point to discuss recommendations for CAPT
systems, by stating that they should integrate what has been found in
research from second language acquisition, and to train pronunciation in
a communicative manner to give context to the learners. They also point
to the problematic area of feedback and advise that systems provide more
easily interpretable feedback with both audio and visual information,
and propose that systems give exercises that are `realistic, varied, and
engaging'. Despite the fact that this article was published in 2002,
this article provides a sound basis in addressing the proper makings of
a successful CAPT system.

In another article by \textcite{eskenazi2009}, we are given a brief
review of technologies in CAPT systems, this time more focused from a
technical perspective. In particular, she gives attention to the
different CAPT system types and provides information on prosody
detection and complete tutoring systems.

She first explains that CAPT systems can be generally split into two
main types: individual error detection and pronunciation assessment. As
indicated, individual error detection systems are more focused on one
particular aspect of the user's speech, such as the phones or pitch,
while pronunciation assessment systems are more designed to represent
how a human would judge a non-native utterance.

Early individual error detection systems, including one of her very own
\textcite{eskenazi1998}, started by using a variety of speech
recognition techniques such as forced alignment or unconstrained speech
recognition. They also worked with a variety of measures to detect the
differences between the individual errors and gold standard. Some of
these measures include hidden Markov model (HMM) based recognition
scoring, a confidence score based system known as Goodness of
Pronunciation (GOP), and Linear Discriminant Analysis (LDA). Each of
these measures were found to somehow detect the users' errors; however
they suffer from issues like low precision or the need for a very
homogeneous sample (e.g.~Japanese speakers).

Here, \textcite{eskenazi2009} makes a point that working to improve
non-native pronunciation is not simply a binary question of native
vs.~non-native; instead the L1 of the system's users must be considered,
as this can greatly affect the evaluation. She also points out that the
level of language learning of the speakers can also impact the metrics
and success of the system as well, and thus an appropriate population
must be selected carefully when building a CAPT system, especially when
considering individual errors.

In her discussion of prosody correction, she points to pivotal works
that have used a variety of manners to address the issue. Some works
include systems that use Pitch Synchronous Overlap and Add (PSOLA) to
resynthesize the prosody of users to help them hear what an appropriate
utterance would sound like. This in particular could be a potentially
effective feedback mechanism to employ in future systems, as it has been
said that imitating one's own voice is the most effective. Other systems
she mentions include systems that use appropriate L2 phonological models
and break prosody down into two levels--- syllable-word and
utterance-phrase, and systems that detect the `liveliness' of a speaker.
However, she does not discuss prosody correction systems in much detail,
which may suggest that there is not as much research in this particular
area as compared to the individual error systems. Regardless, these
works all provide interesting paths to consider in developing a prosody
correction system.

Similar to \textcite{eskenazi2009}, \textcite{chun2008}, presents a
review of various technologies, this time related directly to prosody.
They discuss four main tools in teaching prosody: `visualization of
pitch contours', `multimodal tools', `spectrographic displays' and
`vowel analysis programs'. Citing previous work, it appears that they
suggest that the visualization of pitch contours is the most robust
method of feedback for learners as it is the most intuitive and
non-language specific. Aside from this however, they also discuss the
potential of a multimedia approach used by \textcite{hardison2005} that
integrate both audio and video in a system called \textit{Anvil}.
Following this research, users of this system were able to generalize
their training beyond a sentence level and were able to perform better
at a discourse-level. This again emphasizes the point that prosody
training should put the language in context, which is an important
aspect to consider prosody training, as we know how prosody works in
relation to communication.

They also discuss the two main methods of such prosody systems: one
which utilizes isolated scripted sentences and the other utilizing
imitation. They conclude that neither method is useful for generalizing
to novel methods and suggest that the training should relate to the
ultimate goal. Other information they provide in this article are
prosody models used in previous studies. We see that some previous
studies have focused on utilizing a variety of sentence types to teach
prosody, contrasting \textit{wh}-questions, echo questions, either-or
questions and statements. Like the other articles, the works examined in
\textcite{chun2008} gives us insight on potential ways to improve future
CAPT systems, as we are shown exemplars of potential input and positive
reinforcements in successful types of feedback for the user. They
conclude that in order to create better pronunciation training systems,
we should take advantage of recent technology.

\subsection{Voice conversion}\label{voice-conversion}

Similar to CAPT systems, many researchers have steadily progressed on
building voice conversion systems. However, unlike CAPT systems, VC
systems are not typically designed for uses in language learning and are
more grounded in other speech technology uses such as text-to-speech
synthesis.

To properly frame voice conversion, we take a look at
\cite{mohammadi2017} who presents a recent overview of the subfield.
Following a definition setforth by the authors, voice conversion refers
to the transformation of a speech signal of a \emph{target speaker} to
make it sound similar to a \emph{source speaker} in any chosen fashion
with the utterance still being intact \cite{mohammadi2017}. Some of
these changes can include changes in emotion, accent, or phonation
(whispered/murmured speech). there have been a number of proposed uses
for VC, including the transformation of speaker identity (perhaps for
voice dubbing), personalized TTS systems, and against biometric voice
authentication systems.

Voice conversion often involves a large number of processes, one of
which includes deciding the appropriate type of data. To start, one must
decide whether to have parallel or non-parallel speech data. Parallel
speech data refers to speech data that has source and reference speakers
that say the same utterance, so only the speaker information is
different, while non-parallel data would indicate datasets where the
utterances are not the same.

Even though work in VC has progressed within the last decade or so, it
seems that the task still presents a large challenge for researchers due
to the various nuanced steps and features required to have high quality
voice conversion. This can be witnessed for example, in a shared task
dedicated to voice conversion, appropriately called \emph{The Voice
Conversion Challenge} where many research groups involved in speech
technology around the world have submitted systems in attempts to tackle
the issue. In the second iteration of the challenge
\cite{lorenzo-trueba2018}, the organizers proposed both a parallel and
non-parallel version of the task, both of which were evaluated on
natural and similarity using crowdsourcing.

The type of systems submitted to this year's version of the task
displays the current state of the field and perhaps machine learning
research in general as this year saw a huge increase in the number of
systems using neural networks. However, it does not go without saying
that there were indeed systems that used more traditional statistical
methods, such as Gaussian Mixture Models (GMM) and one of its
variations, differential GMM (DIFFGMM).

In order to evaluate the systems, a group of listeners roughly 300 were
gathered using crowdsourcing. We

Thus, even though not many systems were neural network based, only one
neural network based system was able to outperform the sprocket
GMM-based baseline.

Although we see limitations in the systems presented in The 2018 Voice
Conversion Challenge, there have also been some incredible breakthroughs
in systems set forth by research teams at Google Brain. One such system
involves the Tacotron end-to-end system, which has been proposed to
replace the current set-up of text-to-speech systems by reducing the
amount of components (decoder, vocoder, etc. {[}IS THIS TRUE?{]}) into
one piece. The researchers working on this system have recently revealed
a impressive system that also takes advantage of deep neural networks to
encode speaker characteristics into embeddings, which are then utilized
to transfer style \parencite{wang2018}.

With that said, it is evident that the reason for the success of their
systems is due to the availability of large-scale, high quality data
that many research institutions do not have access to or have funding
for. Thus, it may be a long while before the general public has the
ability to replicate such systems; however it is extremely exciting to
know that there is the possibility.

\subsection{Accent conversion}\label{accent-conversion}

Like voice conversion, accent conversion is dedicated to convert the
speech of a \emph{target speaker} into sounding more like a \emph{source
speaker}. However, accent conversion is specifically focused on morphing
the \emph{accent} of the speech signal, as opposed to sounding directly
like the source speaker. Succinctly stated, ``Accent conversion seeks to
transform second language L2 utterances to appear as if produced with a
native (L1) accent,'' \parencite{aryal2014a}. Accent conversion poses a
further challenge on top of (parallel) voice conversion as the audio of
the source speaker and target speaker is often forced-aligned. This
means that with native and non-native speech, voice conversion would
retain the voice quality and accent of the target speaker
\parencite{aryal2014}.

Finding previous work done in accent conversion has proven to be a
difficult task as there are not many articles available in the area;
this may be because work in voice conversion itself is already a
subfield of speech technology.

With that said, \cite{aryal2014} and other works done by the group of
researchers have made efforts to address the challenge. Throughout their
research, they test a variety of methodologies, including accent
conversion through voice morphing and articulatory synthesis. In the
same work of \cite{aryal2014}, they propose a variation to standard
forced alignment techniques used in voice conversion to pair frames
based on acoustic similarity. To do achieve this, they first dampen
vocal tract differences between the speakers using a method known as
\emph{vocal tract length normalization}.

Finally, in \cite{aryal2015}, they also join in on the rising trend of
utilizing deep neural networks. However, instead of training on solely
audio, they utilize articulatory information for real-time conversion.

Evidentally, an articulatory-based accent conversion system requires
specialized technology that is inaccessible to most users, and thus such
a system cannot be adopted easily.

Aside from the work done by these researchers, not much has been done
since to address accent conversion. Looking at their recent
publications, it seems that they have also halted work in this area, as
they have not published articles in the area since 2016; thus this
leaves a gap between the potential for accent conversion and language
learners.
\cleardoublepage
% \bibliographystyle{apalike-url}
% \renewcommand\bibname{References}
% \nocite{*}
% \bibliography{references}

\printbibliography
\end{document}
